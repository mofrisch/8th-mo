\documentclass{article} 
\usepackage{minted}
\usepackage{amsmath}
\usemintedstyle{pastie}
\parindent0cm
\begin{document} 
File: \texttt{numerics} \\
Copyright (c) 2020, Moritz Frisch\\
All rights reserved.\\
This file may be used according to the BSD 3-Clause License\\
see LICENSE for details\\
 
This is the numerics library, common to all numeric subroutines.
 
\begin{minted}{forth}
ns: nm
\end{minted}
Number of iterations maximally performed in an algorithm to avoid
endless loops, if the procedure converges too slowly.
\begin{minted}{forth}
1000 var, max-iterations
\end{minted}
By default, we do not show all the steps performed.
\begin{minted}{forth}
false var, num-debug
\end{minted}
Width of the exponent display. Your numbers will only look nice,
if your largest exponent is smaller than this.
\begin{minted}{forth}
2 var, exp-size
\end{minted}
Width $w$ of the mantissa. We have three places in front, the sign,
a number and the decimal point, so we have $w-3$ decimal places.
\begin{minted}{forth}
var mantissa-size \ n --
\end{minted}
Set mantissa size.        
\begin{minted}{forth}
: mantissa! \ n --
    dup mantissa-size ! 3 - ## ;
\end{minted}
The default is $\pm1.0000000$.
\begin{minted}{forth}
10 mantissa!
\end{minted}
This is the error bound we normally use, the default is $\epsilon=10^{-16}$.
\begin{minted}{forth}
var epsilon
: eps \ -- n
   epsilon @ ;
: eps! \ n --
   epsilon ! ;
1e-16 eps!
\end{minted}
In a table, we usually print every line. If you want only every $n$-th line,
set this to $n$. The first and last line will always be shown.
\begin{minted}{forth}
1 var, print-lines 
\end{minted}
We use machine numbers by default. If you want big floats, which are 
considerabely slower, you may set e.g. $30$ bigfloats. If you need the
precision, but don't want to display all the places, set \texttt{n\#}
to a convenient value. To avoid problems with rounding or accuracy,
you should use \texttt{big-floats} once and not change it thereafter.
\begin{minted}{forth}
: short-floats \ --
   19 mantissa! 1e-16 eps! ;
\end{minted}
\begin{minted}{forth}
: big-floats \ n --
   dup n:1+ dup n# mantissa! F10 swap n:neg n:^ eps! ;
\end{minted}
\begin{minted}{forth}
short-floats
\end{minted}
Forward declaration of symbols, we often use and like to change.
\begin{minted}{forth}
defer: f
\end{minted}
Return the decimal $\log$ of $n$. Right now $\ln$ does not work for bfloat, 
so we convert to a float. Of course, we don't have the precision we would
expect. Will be fixed in $20.06$, hopefully.
\begin{minted}{forth}
: ld \ n -- n
   n:float n:ln 10 n:ln n:/ ;
\end{minted}
Return the exponent of $n$. For $n > 300$ or so, this is not 
correct because of the $\ln$ error. We could divide successively by $10$ as a
workaround, if the fix is not made soon.
\begin{minted}{forth}
: exponent \ n -- n
   dup 0 > if ld  else
   dup 0 < if n:neg ld else
   then then n:floor n:int ;
\end{minted}
Return the length of the exponent.
\begin{minted}{forth}
: exp-length \ n -- n
   dup 0 = if drop 1 else 
   dup 0 < if n:neg then 
   n:1+ ld n:ceil n:int then ;
\end{minted}
Print the sign of the exponent.
\begin{minted}{forth}
: print-sign \ n -- n
   dup 0 < if "-" . else "+" . then ;
\end{minted}
Print zeroes to fill the exponent.
\begin{minted}{forth}
: print-zeroes \ n -- n
   dup exp-length exp-size @ swap n:- ( "0" . ) swap times ;
\end{minted}
Print the value of the exponent.   
\begin{minted}{forth}
: print-value \ n --
   n:abs . ;
\end{minted}
Print the exponent as $\pm01$.
\begin{minted}{forth}
: print-exponent \ n -- +|-00n
   print-sign print-zeroes print-value ;
\end{minted}
Return the mantissa.
\begin{minted}{forth}
: mantissa \ n exp -- n
   10. swap n:^ n:/ ; 
\end{minted}
Print a number in the format $[-]1.0000000e\pm01$.
\begin{minted}{forth}
: f. \ n --
   mantissa-size @ #>
   dup exponent dup  >r \ n exp
   mantissa . "e" . 
   r> print-exponent ;
\end{minted}
\begin{minted}{forth}
   
\end{minted}
\end{document}

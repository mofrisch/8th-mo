\documentclass{article} 
\usepackage{minted}
\usepackage{amsmath}
\usemintedstyle{pastie}
\parindent0cm
\begin{document} 
File: \texttt{bisection.8th} \\
Copyright (c) 2020, Moritz Frisch\\
All rights reserved.\\
This file may be used according to the BSD 3-Clause License\\
see LICENSE for details\\
 
Find a zero of the given function $f(x)$ in the interval $(a,b)$.
For best performance, we should have $f(a) f(b) < 0$, i.e. the
values should have opposite signs.\\
 
Set \texttt{num-debug} to true, if you want to see the iterations.\\
 
Set \texttt{lines} to something like $5$ or $10$ if you have to much
output. \\
 
If you need more accuracy than the machine accuracy of 15 secure places, 
you can use something like \texttt{30 big-floats}. \\
 
You should also consult the documentation of the numerics library.
 
\begin{minted}{forth}
needs tools
needs numerics
with: nm
with: n
\end{minted}
We store $f(a)$ to not have to evaluate it twice.
\begin{minted}{forth}
var f(a) 
\end{minted}
This is a test function: $f(x)=x^3+4x^2-10$. We evaluate it using
Horner's method: $(x+4)x^2-10$.
\begin{minted}{forth}
: f-test \ n -- n
   dup 4 + swap sqr * 10 - ;
\end{minted}
Calculate the midpoint between $a$ and $b$ using $p=a+\frac{b-a}{2}$,
which is safer against overflow than $\frac{a+b}{2}$.
\begin{minted}{forth}
: midpoint \ a b -- a+(b-a)/2
   tuck - 2 / 4 add-item + ; 
\end{minted}
Exit criterion. Returns true iff $f(p)<\epsilon$ or $|b-p|<\epsilon$.
criterion \ n b p f(p) -- n b p f(p) ?
\begin{minted}{forth}
: bis-criterion \ a b n -- a b n
   row @ 3 a:@ 0~ swap 4 a:@ 0~ nip or ;
\end{minted}
Takes an interval $(a,b)$ and returns a new interval, which is either
$(a,p)$ or $(p,b)$ depending on which the zero lies in. 
next-interval \ a b n --  a p | p b
\begin{minted}{forth}
: bis-algorithm \ a b n -- a p n | p b n
   >r 
   1 add-item
   swap 0 add-item \ b a
   2dup midpoint 2 add-item \ b a p
   dup f 3 add-item \ b a p f(p)
   f(a) @ over \ b a p f(p) f(a) f(p)
   * 0 > if \ b a p f(p)
      f(a) ! nip swap
   else drop rot drop then
   r> ;
\end{minted}
Given an initial interval, that includes at least one zero,
\texttt{bisection} returns an interval with a zero in it.
\begin{minted}{forth}
: bisection \ a b -- a b
   over f f(a) !
   ["a","b","p","f(p)","|b-a|/2"] main ;
\end{minted}
Show the steps
\begin{minted}{forth}
true nm:num-debug !
\end{minted}
\begin{minted}{forth}
' f-test w:is nm:f
' bis-criterion w:is criterion?
' bis-algorithm w:is algorithm
F1. 2. bisection f. space f. cr
;with ;with bye
\end{minted}
\end{document}
